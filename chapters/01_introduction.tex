% !TeX root = ../main.tex
% Add the above to each chapter to make compiling the PDF easier in some editors.
% Add the below not to get bullet point font warnin
\renewcommand\textbullet{\ensuremath{\bullet}}

\chapter{Introduction}\label{chapter:introduction}

The introduction chapter of the thesis will focus on explaining the basic terminology of recommender systems. In addition to that, we will also present the motivation and the hypotheses that we want to validate.


"Recommender Systems are software tools and techniques that provide suggestions for items that are most likely of interest to a particular user." ~\parencite{Ricci2015}

Suggestions and items depend on the field that the recommender system is applied. For example, for the topic of news article recommenders, the aim will most likely be suggesting news to the readers. In the field of job recommenders though, these suggestions can be bidirectional. Meaning that job postings can be suggested to applicants or resumes can be recommended to the human resources team of a company.

Although the history of the recommender systems goes back to mid-1990s ~\parencite{PARK201210059}, the real boom happened after e-commerce services became mainstream \cite{smith2017two}. Since there were too many items to choose from for users, such service was needed. Users of websites were becoming overloaded with the information, and the developers of recommender systems had the aim of reducing the information to be only relevant to users. 


As mentioned in the last paragraph, recommender systems have the general function of suggesting items to users. However, why do recommender systems get developed? What kind of benefit do they have for both companies and users?

First of all, recommender systems increase the number of items sold  ~\parencite{Ricci2015}.  Also, most recommenders suggest personalized results, which means that users will see content that fits their desires. These recommendations will also increase users buying more items.

Recommenders also increase the coverage of items that user see. Coverage denotes the number of recommended unique items divided by the number of all items. Therefore, users can interact with items that they would not even see without recommenders, that improves the chances of buying more items.

Another essential point is increasing user satisfaction. This function of recommenders is the foundation of the thesis at hand. Unfortunately, most of the researchers do not take into account that the user satisfaction does not solely depend on simple evaluation metrics like accuracy, precision or recall but it can also depend on privacy, data security, diversity, serendipity, labeling, and presentation ~\parencite{Beel2016}. When recommender systems take those factors into account, they increase user satisfaction.


The recommendation performance is mainly evaluated in terms of accuracy (i.e., the difference between actual and predicted rating). Concentrating on the accuracy might not be a desired goal of the talent/project recommendations (e.g., merely recommending a \textit{top-N} list of talents that best match the requirements might result in similar talents whose skills only vary in a small extent). Besides the desired accuracy, there might be other properties necessary for adequate recommendations (e.g., privacy, data security, diversity, serendipity, labeling, and presentation, and group recommendation).


Recommender systems can be applied to a various set of fields like entertainment, content creation, e-commerce, service sector, and social. However, we chose a specific use-case, which is the suggestion of talents to projects to be managed by the recruiters of the companies.


To be able to develop a solution on this topic, we used two datasets: one from the website \textit{Freelancer.com} and another one from the company \textit{Motius Gmbh}. These datasets are presented in the section \ref{section:datasets}.


\section{Motivation}\label{section:motivation}

This thesis aims to find answers to three main questions. The first question is if high accuracy guarantees high user satisfaction. The second one is an extension to the first question and asks if diversity affects user satisfaction positively. The last question asks if a feedback loop increases user satisfaction eventually.


\section{Contribution}

To answer the questions that we asked in section \ref{section:motivation}, we first developed models and algorithms that generate predictions. Then, we implemented these models and algorithms into a dashboard. Lastly, we executed offline, online evaluation rounds, and user studies to validate our hypotheses that were formulated as questions. After performing the previous steps, we can finally answer the questions that we asked in section \ref{section:motivation}. 

First of all, we saw that high accuracy does not always result in high user satisfaction. This case is because other properties also affect user satisfaction, and having a marginal increase in the accuracy may result in a reduction of other properties, that also may decrease user satisfaction. These properties may vary from setting to setting and use-case to use-case. 

We found out that diversity is an essential factor for our case, which is recommending a group of talents to a project. Although accuracy and diversity are inversely proportional, a small increase in diversity led to a slightly higher user satisfaction. However, a further marginal increase in diversity resulted in a decrease in user satisfaction. 

Lastly, we validated if a feedback loop increases user satisfaction. Although the initial implementation and training with a feedback loop do not increase user satisfaction, the subjects of the user study saw a positive effect, after giving more and more feedback. After inserting more than a thousand feedback and retraining with the resulting data, users were more satisfied.

We conclude that user satisfaction in recommender systems can be increased if the developers take user feedback into account and not only focus on algorithmic accuracy.