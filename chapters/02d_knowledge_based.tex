% !TeX root = ../main.tex
% Add the above to each chapter to make compiling the PDF easier in some editors.

\subsection{Knowledge-Based Recommender Systems}\label{chapter:knowledge_based}

Knowledge-based recommenders are the third most popular type of recommender systems \cite{burke2000knowledge}. In this type of recommender, users are explicitly asked to enter criteria for what kind of items they want to see. This information is compared with the items, and a recommendation list is returned. More information can be found in the resources \cite{burke2000knowledge} and \cite{felfernig2015constraint}.

\subsection{Hybrid Recommender Systems}

Finally, to overcome certain limitations of content-based and collaborative filtering methods, hybrid recommendation approaches combine characteristics of both types of methods. Content-based and collaborative filtering methods can be combined "in various ways, for instance, by merging their individual predictions into a single, more robust prediction" \cite{desrosiers2011comprehensive}, or by adding content information into a collaborative filtering model. "Several studies have shown hybrid recommendation approaches to provide more accurate recommendations than pure content-based or collaborative methods, especially when few ratings are available" \cite{desrosiers2011comprehensive},  which is called the cold-start problem.


\section{Types of input in recommender systems}

Recommender systems can receive different types of input. The straightforward one would be the explicit feedback that represents the actual ratings that users give to the items. This rating could be a star rating or a thumbs up/down action. The second is implicit feedback, which is harder to collect. It represents clicks, browsing history, mouse movements, and other ways of input that the users do not send explicitly.
