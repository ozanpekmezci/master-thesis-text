% !TeX root = ../main.tex
% Add the above to each chapter to make compiling the PDF easier in some editors.

\section{Collaborative Filtering}\label{section:collaborative_filtering}

Collaborative filtering recommender system methods produce user-specific recommendations of items based on patterns of ratings or hiring data without the need for content information about either items or users \cite{desrosiers2011comprehensive}.

Collaborative filtering recommenders need interactions between users and items. Then, these recommenders can establish recommendations with two different approaches: the neighborhood approach and latent factor models. Neighborhood methods focus on relationships between items and users and can function by checking the similarity between users or items.  Latent factor models, such as matrix factorization transfer the user-item matrix into a different space. By doing this operation, it is possible to fill the gaps in the matrix, which correspond to the predictions.

Since we have not used collaborative filtering in the implementation part of this thesis, we keep this section short. Readers who are interested in this topic are advised to check the resources \cite{koren2015advances} and \cite{wang2015collaborative}.
