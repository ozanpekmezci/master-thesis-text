% !TeX root = ../main.tex
% Add the above to each chapter to make compiling the PDF easier in some editors.

\subsection{Collaborative Filtering}\label{section:collaborative_filtering}

Rather than looking for content information, collaborative filtering approaches use the rating information of other users and items in the system. The main idea is that the rating of a target user for a new item is similar to that of another user. Similarly, the target user probably evaluates two elements in a similar way if other users have given similar ratings to these two elements. Collaborative approaches overcome some of the weaknesses of content-based methods. With collaborative filtering, the recommender system can still suggest items to users, even though the system does not know about the new user or item. Furthermore, collaborative recommendations are based on the ratings of items as evaluated by peers instead of relying on content. "Finally, unlike content-based systems, collaborative filtering ones can recommend items with very different content, as long as other users have already shown interest for these different items" \cite{desrosiers2011comprehensive}.

Collaborative filtering recommender system methods produce user-specific recommendations of items based on patterns of ratings or hiring data without the need for content information about either items or users \cite{desrosiers2011comprehensive}.

Collaborative filtering recommenders need interactions between users and items. Then, these recommenders can establish recommendations with two different approaches: the neighborhood approach and latent factor models. Neighborhood methods focus on relationships between items and users and can function by checking the similarity between users or items.  Latent factor models, such as matrix factorization transfer the user-item matrix into a different space. By doing this operation, it is possible to fill the gaps in the matrix, which correspond to the predictions.

Since we have not used collaborative filtering in the implementation part of this thesis, we keep this section short. Readers who are interested in this topic are advised to check the resources \cite{koren2015advances} and \cite{wang2015collaborative}.
