% !TeX root = ../main.tex
% Add the above to each chapter to make compiling the PDF easier in some editors.

\chapter{Content-Based Filtering}\label{chapter:content_based_filtering}

\section{Introduction}

[kopi peyst from semantics aware cbf, 4]

Content-based recommender systems (CBRSs) rely on item and user descriptions (content) to build item representations and user profiles to suggest items similar to those a target user already liked in the past. The basic process of producing content-based recommendations consists in matching up the attributes of the target user profile, in which preferences and interests are stored, with the attributes of the items. The result is a relevance score that predicts the target user’s level of interest in those items. Usually, attributes for describing an item are features extracted from metadata associated to that item, or textual features extracted directly from the item description. The content extracted from metadata is often too short and not sufficient to correctly define the user interests, while the use of textual features involves a number of complications when learning a user profile due to natural language ambiguity. Polysemy, synonymy, multi-word expressions, named entity recognition and disambiguation are inherent problems of traditional keyword-based profiles, which are not able to go beyond the usage of lexical/syntactic structures to infer the user interest in topics.

The ever increasing interest in semantic technologies and the availability of several open knowledge sources, such as Wikipedia, DBpedia, Freebase, and BabelNet have fueled recent progress in the field of CBRSs. Novel research works have introduced semantic techniques that shift from a keyword-based to a concept- based representation of items and user profiles. These observations make very relevant the integration of proper techniques for deep content analytics borrowed from Natural Language Processing (NLP) and Semantic Technologies, which is one of the most innovative lines of research in semantic recommender systems [61].

We roughly classify semantic techniques into top-down and bottom-up approaches. Top-down approaches rely on the integration of external knowledge, such as machine readable dictionaries, taxonomies (or IS-A hierarchies), thesauri or ontologies (with or without value restrictions and logical constraints), for annotating items and representing user profiles in order to capture the semantics of the target user information needs. The main motivation behind top-down approaches is the challenge of providing recommender systems with the linguistic knowledge and common sense knowledge, as well as the cultural background which characterize the human ability of interpreting documents expressed in natural language and reasoning on their meaning.

On the other side, bottom-up approaches exploit the so-called geometric metaphor of meaning to represent complex syntagmatic and paradigmatic relations between words in high-dimensional vector spaces. According to this metaphor, each word (and each document as well) can be represented as a point in a vector space. The peculiarity of these models is that the representation is learned by analyzing the context in which the word is used, in a way that terms (or documents) similar to each other are close in the space. For this reason bottom-up approaches are also called distributional models. One of the great virtues of these approaches is that they are able to induce the semantics of terms by analyzing their use in large corpora of textual documents using unsupervised mechanisms, as evidenced by the recent advances of machine translation techniques [52, 83].

This chapter describes a variety of semantic approaches, both top-down and bottom-up, and shows how to leverage them to build a new generation of semantic CBRSs that we call semantics-aware content-based recommender systems.

xd

This section reports an overview of the basic principles for building CBRSs, the main techniques for representing items, learning user profiles and providing recommendations. The most important limitations of CBRSs are also discussed, while the semantic techniques useful to tackle those limitations are introduced in the next sections.

The high level architecture of a content-based recommender system is depicted in Fig. 4.1. The recommendation process is performed in three steps, each of which is handled by a separate component:

[Figure 4.1 ya da benzeri ekle]

\begin{itemize}
\item CONTENT ANALYZER—When information has no structure (e.g. text), some kind of pre-processing step is needed to extract structured relevant information. The main responsibility of the component is to represent the content of items (e.g. documents, Web pages, news, product descriptions, etc.) coming from information sources in a form suitable for the next processing steps. Data items are analyzed by feature extraction techniques in order to shift item representation from the original information space to the target one (e.g. Web pages represented as keyword vectors). This representation is the input to the PROFILE LEARNER and FILTERING COMPONENT;
\item PROFILE LEARNER—This module collects data representative of the user preferences and tries to generalize this data, in order to construct the user profile. Usually, the generalization strategy is realized through machine learning techniques [86], which are able to infer a model of user interests starting from items liked or disliked in the past. For instance, the PROFILE LEARNER of a Web page recommender can implement a relevance feedback method [113] in which the learning technique combines vectors of positive and negative examples into a prototype vector representing the user profile. Training examples are Web pages on which a positive or negative feedback has been provided by the user;
\item FILTERING COMPONENT—This module exploits the user profile to suggest relevant items by matching the profile representation against that of items to be recommended. The result is a binary or continuous relevance judgment (computed using some similarity metrics [57]), the latter case resulting in a ranked list of potentially interesting items. In the above mentioned example, the matching is realized by computing the cosine similarity between the prototype vector and the item vectors.
\end{itemize}

The first step of the recommendation process is the one performed by the CONTENT ANALYZER, that usually borrows techniques from Information Retrieval systems [6, 118]. Item descriptions coming from Information Source are processed by the CONTENT ANALYZER, that extracts features (keywords, n-grams, concepts, . . . ) from unstructured text to produce a structured item representation, stored in the repository Represented Items.


In order to construct and update the profile of the active user ua (user for which recommendations must be provided) her reactions to items are collected in some way and recorded in the repository Feedback. These reactions, called anno- tations [51] or feedback, together with the related item descriptions, are exploited during the process of learning a model useful to predict the actual relevance of newly presented items. Users can also explicitly define their areas of interest as an initial profile without providing any feedback. Typically, it is possible to distinguish between two kinds of relevance feedback: positive information (inferring features liked by the user) and negative information (i.e., inferring features the user is not interested in [58]). Two different techniques can be adopted for recording user’s feedback. When a system requires the user to explicitly evaluate items, this technique is usually referred to as “explicit feedback”; the other technique, called “implicit feedback”, does not require any active user involvement, in the sense that feedback is derived from monitoring and analyzing user’s activities. Explicit evaluations indicate how relevant or interesting an item is to the user [111]. Explicit feedback has the advantage of simplicity, albeit the adoption of numeric/symbolic scales increases the cognitive load on the user, and may not be adequate for catching user’s feeling about items. Implicit feedback methods are based on assigning a relevance score to specific user actions on an item, such as saving, discarding, printing, bookmarking, etc. The main advantage is that they do not require a direct user involvement, even though biasing is likely to occur, e.g. interruption of phone calls while reading.

In order to build the profile of the active user ua, the training set TRa for ua must be defined. TRa is a set of pairs <Ik, rk> i, where rk is the rating provided by ua on the item representation Ik. Given a set of item representation labeled with ratings, the PROFILE LEARNER applies supervised learning algorithms to generate a predictive model—the user profile—which is usually stored in a profile repository for later use by the FILTERING COMPONENT. After the user profile has been learned, the FILTERING COMPONENT predicts whether a new item is likely to be of interest for the active user, by comparing features in the item representation to those in the representation of user preferences (stored in the user profile).

User tastes usually change in time, therefore up-to-date information must be maintained and provided to the PROFILE LEARNER in order to automatically update the user profile. Further feedback is gathered on generated recommendations by letting users state their satisfaction or dissatisfaction with items in La. After gathering that feedback, the learning process is performed again on the new training set, and the resulting profile is adapted to the updated user interests. The iteration of the feedback-learning cycle over time enables the system to take into account the dynamic nature of user preferences.

\subsection{Text to Vector Space Model}

[4.2.1 tf-idf, cosine similarity falan]

\subsection{Methods for Learning User Profile}

[4.2.2.1 eger naive bayes kullanirsan 4.2.2.1, 4.2.2.2 bilmiyorum, 4.2.2.3 kesin]

\subsection{Advantages and Drawbacks of Content-Based Filtering}

The adoption of the content-based recommendation paradigm has several advan- tages when compared to the collaborative one:

\begin{itemize}
	\item USER INDEPENDENCE—Content-based recommenders exploit solely ratings provided by the active user to build her own profile. Instead, collaborative filtering methods need ratings from other users in order to find the “nearest neighbors” of the active user, i.e., users that have similar tastes since they rated the same items similarly. Then, only the items that are most liked by the neighbors of the active user will be recommended;
	\item TRANSPARENCY—Explanations on how the recommender system works can be provided by explicitly listing content features or descriptions that caused an item to occur in the list of recommendations. Those features are indicators to consult in order to decide whether to trust a recommendation. Conversely, collaborative systems are black boxes since the only explanation for an item recommendation is that unknown users with similar tastes liked that item;
	\item NEW ITEM—Content-based recommenders are capable of recommending items not yet rated by any user. As a consequence, they do not suffer from the first-rater problem, which affects collaborative recommenders which rely solely on users’ preferences to make recommendations. Therefore, until the new item is rated by a substantial number of users, the system would not be able to recommend it.
\end{itemize}

Nonetheless, content-based systems have several shortcomings:

\begin{itemize}
	\item LIMITED CONTENT ANALYSIS—Content-based techniques have a natural limit in the number and type of features that are associated, whether automatically or manually, with the objects they recommend. Domain knowledge is often needed, e.g., for movie recommendations the system needs to know the actors and directors, and sometimes, domain ontologies are also needed. No content- based recommendation system can provide suitable suggestions if the analyzed content does not contain enough information to discriminate items the user likes from items the user does not like. Some representations capture only certain aspects of the content, but there are many others that would influence a user’s experience. For instance, often there is not enough information in the word frequency to model the user interests in jokes or poems, while techniques for affective computing would be most appropriate. Again, for Web pages, feature extraction techniques from text completely ignore aesthetic qualities and additional multimedia information. Furthermore, CBRSs based on a string matching approach suffer from problems of:
	\begin{itemize}
		\item POLYSEMY, the presence of multiple meanings for one word;
		\item SYNONYMY, multiple words with the same meaning;
		\item MULTI-WORD EXPRESSIONS, the difficulty to assign the correct properties to
		a sequence of two or more words whose properties are not predictable from
		the properties of the individual words;
		\item ENTITY IDENTIFICATION or NAMED ENTITY RECOGNITION, the difficulty
		to locate and classify elements in text into pre-defined categories such as the names of persons, organizations, locations, expressions of times, quantities, monetary values, etc.
		\item ENTITY LINKING or NAMED ENTITY DISAMBIGUATION, the difficulty of determining the identity (often called the reference) of entities mentioned in text.
	\end{itemize}
	\item OVER-SPECIALIZATION—Content-based recommenders have no inherent method for finding something unexpected. The system suggests items whose scores are high when matched against the user profile, hence the user is going to be recommended items similar to those already rated. This drawback is also called lack of serendipity problem to highlight the tendency of the content-based systems to produce recommendations with a limited degree of novelty. To give an example, when a user has only rated movies directed by Stanley Kubrick, she will be recommended just that kind of movies. A “perfect” content-based technique would rarely find anything novel, limiting the range of applications for which it would be useful.
	\item NEW USER—Enough ratings have to be collected before a content-based rec- ommender system can really understand user preferences and provide accurate recommendations. Therefore, when few ratings are available, as for a new user, the system will not be able to provide reliable recommendations.
\end{itemize}

[4.3 Eger textten anlam cikarma falan yaparsan eklersin]
[4.4 Eger textten anlam cikarma falan yaparsan eklersin, bir oncekinin daha basiti ama iste insan olmadan halletme gibi]
[4.5 ayni sekilde]

\section{Conclusion}
[Biraz 4.6 dan alabilirsin ama hepsi alakali degil]

