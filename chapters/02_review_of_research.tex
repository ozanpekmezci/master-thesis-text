% !TeX root = ../main.tex
% Add the above to each chapter to make compiling the PDF easier in some editors.

\chapter{Review of Literature and Research}\label{chapter:review_of_research}

\section{Types of Recommender Systems}
[kopi peyst]
Item recommendation approaches can be divided in two broad categories: per- sonalized and non-personalized. Among the personalized approaches are content- based and collaborative filtering methods, as well as hybrid techniques combining these two types of methods. The general principle of content-based (or cognitive) methods [4, 8, 42, 54] is to identify the common characteristics of items that have received a favorable rating from a user, and then recommend to this user new items that share these characteristics. Recommender systems based purely on content generally suffer from the problems of limited content analysis and over- specialization [63]. Limited content analysis occurs when the system has a limited amount of information on its users or the content of its items. For instance, privacy issues might refrain a user from providing personal information, or the precise content of items may be difficult or costly to obtain for some types of items, such as music or images. Another problem is that the content of an item is often insufficient to determine its quality. Over-specialization, on the other hand, is a side effect of the way in which content-based systems recommend new items, where the predicted rating of a user for an item is high if this item is similar to the ones liked by this user. For example, in a movie recommendation application, the system may recommend to a user a movie of the same genre or having the same actors as movies already seen by this user. Because of this, the system may fail to recommend items that are different but still interesting to the user.

Instead of depending on content information, collaborative (or social) filtering approaches use the rating information of other users and items in the system. The key idea is that the rating of a target user for a new item is likely to be similar to that of another user, if both users have rated other items in a similar way. Likewise, the target user is likely to rate two items in a similar fashion, if other users have given similar ratings to these two items. Collaborative approaches overcome some of the limitations of content-based ones. For instance, items for which the content is not available or difficult to obtain can still be recommended to users through the feedback of other users. Furthermore, collaborative recommendations are based on the quality of items as evaluated by peers, instead of relying on content that may be a bad indicator of quality. Finally, unlike content-based systems, collaborative filtering ones can recommend items with very different content, as long as other users have already shown interest for these different items.

Collaborative filtering approaches can be grouped in the two general classes of neighborhood and model-based methods. In neighborhood-based (memory-based [10] or heuristic-based [2]) collaborative filtering [14, 15, 27, 39, 44, 48, 57, 59, 63], the user-item ratings stored in the system are directly used to predict ratings for new items. This can be done in two ways known as user-based or item-based recommendation. User-based systems, such as GroupLens [39], Bellcore video [27], and Ringo [63], evaluate the interest of a target user for an item using the ratings for this item by other users, called neighbors, that have similar rating patterns. The neighbors of the target user are typically the users whose ratings are most correlated to the target user’s ratings. Item-based approaches [15, 44, 59], on the other hand, predict the rating of a user for an item based on the ratings of the user for similar items. In such approaches, two items are similar if several users of the system have rated these items in a similar fashion.

In contrast to neighborhood-based systems, which use the stored ratings directly in the prediction, model-based approaches use these ratings to learn a predictive model. Salient characteristics of users and items are captured by a set of model parameters, which are learned from training data and later used to predict new ratings. Model-based approaches for the task of recommending items are numerous and include Bayesian Clustering [10], Latent Semantic Analysis [28], Latent Dirich- let Allocation [9], Maximum Entropy [72], Boltzmann Machines [58], Support Vector Machines [23], and Singular Value Decomposition [6, 40, 53, 68, 69]. A survey of state-of-the-art model-based methods can be found in Chap. 3 of this book.

Finally, to overcome certain limitations of content-based and collaborative filtering methods, hybrid recommendation approaches combine characteristics of both types of methods. Content-based and collaborative filtering methods can be combined in various ways, for instance, by merging their individual predictions into a single, more robust prediction [8, 55], or by adding content information into a collaborative filtering model [1, 3, 51, 65, 71]. Several studies have shown hybrid recommendation approaches to provide more accurate recommendations than pure content-based or collaborative methods, especially when few ratings are available [2].

\section{Neighborhood-Based Approach}
[kopi peyst]
While recent investigations show state-of-the-art model-based approaches superior to neighborhood ones in the task of predicting ratings [40, 67], there is also an emerging understanding that good prediction accuracy alone does not guarantee users an effective and satisfying experience.

Model-based approaches excel at characterizing the preferences of a user with latent factors. For example, in a movie recommender system, such methods may determine that a given user is a fan of movies that are both funny and romantic, without having to actually define the notions “funny” and “romantic”. This system would be able to recommend to the user a romantic comedy that may not have been known to this user. However, it may be difficult for this system to recommend a movie that does not quite fit this high-level genre, for instance, a funny parody of horror movies. Neighborhood approaches, on the other hand, capture local associations in the data. Consequently, it is possible for a movie recommender system based on this type of approach to recommend the user a movie very different from his usual taste or a movie that is not well known (e.g. repertoire film), if one of his closest neighbors has given it a strong rating. This recommendation may not be a guaranteed success, as would be a romantic comedy, but it may help the user discover a whole new genre or a new favorite actor/director. -> Model based suck at serendipity

\begin{itemize}
	\item Simplicity: Neighborhood-based methods are intuitive and relatively simple to implement. In their simplest form, only one parameter (the number of neighbors used in the prediction) requires tuning.
	\item  Justifiability: Such methods also provide a concise and intuitive justification for the computed predictions. For example, in item-based recommendation, the list of neighbor items, as well as the ratings given by the user to these items, can be presented to the user as a justification for the recommendation. This can help the user better understand the recommendation and its relevance, and could serve as basis for an interactive system where users can select the neighbors for which a greater importance should be given in the recommendation [6].
	\item Efficiency: One of the strong points of neighborhood-based systems are their efficiency. Unlike most model-based systems, they require no costly training phases, which need to be carried at frequent intervals in large commercial applications. These systems may require pre-computing nearest neighbors in an offline step, which is typically much cheaper than model training, providing near instantaneous recommendations. Moreover, storing these nearest neighbors requires very little memory, making such approaches scalable to applications having millions of users and items.
	\item Stability: Another useful property of recommender systems based on this approach is that they are little affected by the constant addition of users, items and ratings, which are typically observed in large commercial applications. For instance, once item similarities have been computed, an item-based system can readily make recommendations to new users, without having to re-train the system. Moreover, once a few ratings have been entered for a new item, only the similarities between this item and the ones already in the system need to be computed.
\end{itemize}

While neighborhood-based methods have gained popularity due to these advan- tages, they are also known to suffer from the problem of limited coverage, which causes some items to be never recommended. Also, traditional methods of this category are known to be more sensitive to the sparseness of ratings and the cold-start problem, where the system has only a few ratings, or no rating at all, for new users and items. Section 2.5 presents more advanced neighborhood-based techniques that can overcome these problems.

