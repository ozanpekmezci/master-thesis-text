% !TeX root = ../main.tex
% Add the above to each chapter to make compiling the PDF easier in some editors.

\chapter{Review of Literature and Research}\label{chapter:review_of_research}

This chapter of the thesis aims to create a theoretical background of the topics that were demonstrated later in this thesis.

\section{Types of Recommender Systems}

Item recommendation approaches can be divided in two main categories: personalized and non-personalized. Among the personalized approaches are content- based and collaborative filtering methods, as well as hybrid techniques combining these two types of methods. The general principle of content-based methods is to identify the common characteristics of items that have received a favorable rating from a user, and then suggest to this user new items that share these characteristics. Recommender systems based only on content generally suffer from the problems of limited content analysis and over-specialization \cite{shardanand1995social}. Limited content analysis occurs when the system has a limited amount of information on its users or the content of its items. This means the content of an item is often insufficient to determine its quality. Over-specialization, on the other hand, is the process of recommending similar items to other items that the user rated positively. Because of this, the system may fail to recommend items that are different but still interesting to the user \cite{desrosiers2011comprehensive}.


Rather than looking for content information, collaborative filtering approaches use the rating information of other users and items in the system. The main idea is that the rating of a target user for a new item is similar to that of another user. Similarly, the target user probably evaluates two elements in a similar way, if other users have given similar ratings to these two elements. Collaborative approaches overcome some of the limitations of content-based methods. With collaborative filtering, the recommender system can still suggest items to users, even though the system doesn't know about the new user or item. Furthermore, collaborative recommendations are based on the ratings of items as evaluated by peers, instead of relying on content. Finally, unlike content-based systems, collaborative filtering ones can recommend items with very different content, as long as other users have already shown interest for these different items \cite{desrosiers2011comprehensive}.

Finally, to overcome certain limitations of content-based and collaborative filtering methods, hybrid recommendation approaches combine characteristics of both types of methods. Content-based and collaborative filtering methods can be combined in various ways, for instance, by merging their individual predictions into a single, more robust prediction, or by adding content information into a collaborative filtering model. Several studies have shown hybrid recommendation approaches to provide more accurate recommendations than pure content-based or collaborative methods, especially when few ratings are available, which is called the cold-start problem \cite{desrosiers2011comprehensive}.

\section{Types of input in recommender systems}

Recommender systems can receive different types of input. The straightforward one would be the explicit feedback that represents the actual ratings that users give to the items. This could be a star rating or a thumbs up/down action. The second is the implicit feedback, which is harder to collect. It represents clicks, browsing history, mouse movements and other ways of input that the users don't send explicitly.

