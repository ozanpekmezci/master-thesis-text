\chapter{\abstractname}

Recommendation systems have an essential aim of increasing user satisfaction. However, many developers of these systems do not take into account that user satisfaction does not only depend on simple evaluation metrics like accuracy, but it can also depend on privacy, data security, diversity, serendipity, labeling, and presentation.

This work aimed to find answers to three main questions. The first question is whether high accuracy guarantees high user satisfaction. The second is an extension of the first question and asks if diversity has a positive impact on user satisfaction. The last question asks if a feedback loop increases user satisfaction.

To answer these questions, we have developed models and algorithms to generate predictions. Then we implemented these models and algorithms in a dashboard. Most recently, we conducted offline and online evaluation rounds and user studies to validate our hypothesized questions. First, we have seen that high accuracy does not always result in high user satisfaction. This is because other properties can also affect user satisfaction and a slight increase in accuracy can lead to a reduction in other properties that can also reduce user satisfaction. These properties can vary from setting to setting and from application to application. We have found that diversity is a critical factor in our use-case, which is the recommendation of a group of talents for a project. Although accuracy and diversity are inversely proportional, a slight increase in diversity led to slightly higher user satisfaction. However, a significant increase in diversity led to a decline in user satisfaction. Last, we checked if a feedback loop increases user satisfaction. Although initial implementation and training with a feedback loop does not increase user satisfaction, subjects in the user study reported more positive feedback after more and more feedback. After inserting more than a thousand feedbacks and re-training with the resulting data, users were happier with the results. We have concluded that user satisfaction in recommender systems can be increased if developers take user feedback into account and not solely focus on algorithmic accuracy.

It is hoped this thesis will inform researchers and developers of recommender systems about increasing user satisfaction.

Keywords: recommendation systems, accuracy, diversity, evaluation properties, group recommendations, job recommender system


